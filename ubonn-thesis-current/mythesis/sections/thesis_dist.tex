%==============================================================================
\chapter{Quantification of the Difference}
\label{chap:quantifing}
%==============================================================================

Having shown explicitly the discrepancy between the pure and the thermal 
descriptions as well as where it is most significant, the author will then
quantify the difference. Related definitions of distances in quantum 
mechanics are introduced in \cref{chap:distappd}.
%In this chapter, it will be explained why the difference between the two cases 
%is expected to be negatively correlated to the mass of the black hole, then 
%the difference will be evaluated. 

%------------------------------------------------------------------------------
\section{The Canonical State} %Popescu--Winter Theorem
\label{sec:arbitrary-canonical}
%------------------------------------------------------------------------------

It has been shown in \cite{Popescu2006} that a small subsystem in a large, 
isolated system which is subject to a \emph{constraint}, is expected to be very 
close to a \emph{canonical state} of it, which reduces to the usual 
thermodynamic \emph{canonical ensemble} when the total system is under an 
\emph{energy constraint}.

Denoted by $U$, the large system has all its possible pure states in Hilbert 
space $\mscrH_U$ which has a finite dimension $d_R = \dim \mscrH_U$. A global 
constraint $R$ is also imposed, which restricts the physical Hilbert space of 
$U$ to $\mscrH_R \subseteq \mscrH_U$. The \emph{equiprobable state} of $U$ is
\begin{equation}
\mscrE_R \coloneqq d_R^{-1} \Bbbone_R,
\end{equation}
where $\Bbbone_R$ is the identity operator on $\mscrH_R$. An energy constraint, 
for instance, reads
\begin{equation}
\abr{H_U} = E_R,
\label{eq:ene-constraint}
\end{equation}
so that a state $\Ket{\alpha}$ in $\mscrH_R^{\rbr{\text{E}}}$ satisfies
\begin{equation}
E_R = \Braket{\alpha | H_U | \alpha} = \sum_E \vbr{\Braket{E | \alpha}}^2 E,
\end{equation}
where $\Ket{E}$ is an energy eigenstate of $U$ with eigenvalue $E$.

The subsystem in concern is named $S$, with \emph{all possible} states in the
Hilbert space $\mscrH_S$; the rest of $U$ is called the \emph{environment} and 
denoted by $E$, the pure states of which lie in $\mscrH_E$. One has
\begin{equation}
\mscrH_S \otimes \mscrH_E = \mscrH_U \supseteq \mscrH_R,\qquad
H_U = H_S + H_E + H_\text{int},
\end{equation}
where $H_S$, $H_E$ and $H_\text{int}$ are the system, environment and 
interaction Hamiltonian, respectively. 

The \emph{canonical state} of $S$ is defined as
\begin{equation}
{\Omega}_S \coloneqq \tr_E {\mscrE}_R,
\end{equation}
where the trace is taken over all the degrees of freedom in the environment. 
When the energy constraint \cref{eq:ene-constraint} is used, it can be shown 
that $\Omega_S^{(E)}$ is a canonical ensemble\footnote{A derivation of 
\cref{eq:canonical-state-def} in classical statistical physics can be found in 
\cite[sec.~28]{Landau1980}.}
\begin{equation}
\Omega_S^{(E)} \propto \ee^{- H_S/T_\text{m}}
= \sum_{E_S} \ee^{-E_S/T_\text{m}} \Ket{E_S}\Bra{E_S},
\label{eq:canonical-state-def}
\end{equation}
where $T_\text{m}$ can be identified with a temperature, and $\Ket{E_S}$ is an
eigenstate of the system with eigenvalue $E$. In this case it reduces to 
the traditional thermodynamic statistical physics.

%To state the theorems, one additional definition is needed: 
The environment also has an \emph{effective} dimension
\begin{equation}
d_E^\text{eff} \coloneqq \rbr{\tr \Omega_E^2}^{-1} \geq d_R/d_S,
\label{eq:def-effective-dim}
\end{equation}
where ${\Omega}_E = \tr_S {\mscrE}_R$. When no constraint is enforced, so that
$\mscrH_S \otimes \mscrH_E \equiv \mscrH_U = \mscrH_R$, 
\cref{eq:def-effective-dim} reduces to
\begin{equation}
d_E^\text{eff} = d_R/d_S = d_E.
\end{equation}
Detailed discussions about $d_E^\text{eff}$ can be found in \cite{Popescu2006}.

Equipped with all the definitions above, one picks an arbitrary pure state 
$\Ket{\phi}\in\mscrH_R$ and denote the \emph{reduced state} of $S$ by
\begin{equation}
\rfun{\rho_S}{\phi} = \tr_E \Ket{\phi}\Bra{\phi}
\end{equation}
Then a lemma states that the average \emph{trace distance}\footnote{See 
\cref{sec:trace-dist}.} between $\rho_S$ and $\Omega_S$ is very small in terms 
of the ratio between $d_S$ and $d_S/d_E^\text{eff}$, i.e.\
\begin{equation}
\abr{\rfun{T}{\rfun{\rho_S}{\phi}, {\Omega}_S}} \le \frac{1}{2}
\sqrt{\frac{d_S}{d_E^\text{eff}}}.
\end{equation}
In a typical division where $d_S/d_E$ is small, this average distance will also 
be tiny.

More over, the main theorem asserts that those ${\rho}_S$'s which are close to 
${\Omega}_S$ dominate; the probability of a large deviation is exponentially 
small with respect to the deviation. For an arbitrary $\epsilon > 0$, the 
theorem states that
\begin{equation}
\frac{\sfun{V}{\cbr{\Ket{\phi} \in \mscrH_R |
\rfun{T}{\rfun{\rho_S}{\phi}, \Omega_S} \geq \eta}}}
{\sfun{V}{\cbr{\Ket{\phi} \in \mscrH_R}}} \leq \eta',
\label{eq:main-theorem-Popescu}
\end{equation}
where
\begin{equation}
\eta = \epsilon + \frac{1}{2}\sqrt{\frac{d_S}{d_E^\text{eff}}}; \qquad
\eta' = 4\rfun{\exp}{-C d_R \epsilon^2},\quad C = \frac{2}{9\pp^3}.
\label{eq:main-theorem-supp}
\end{equation}
To understand the theorem, first note that the left-hand side of 
\cref{eq:main-theorem-Popescu} is a probability measure. More over, one can 
choose $\epsilon = d_R^{-1/3}$ as well, so that
\begin{equation}
\eta = \epsilon + \frac{1}{2}\sqrt{\frac{d_S}{d_E^\text{eff}}} \gtrsim 
d_R^{-1/3};\qquad
\eta' = 4\rfun{\exp}{-C d_R \epsilon^2} = 4\rfun{\exp}{-C d_R^{+1/3}},
\end{equation}
where $d_E^\text{eff} \gg d_S$ is also assumed. In this case, the probability 
of the deviation greater than $d_R^{-1/3}$ is smaller than an 
\emph{exponential} of $d_R^{+1/3}$.

Further explanations and proofs of the lemma and the theorem can 
be found in \cite{Popescu2006,Popescu2007}.

%------------------------------------------------------------------------------
\section{Distance between the Pure and Mixed States}
\label{sec:dist-radiation}
%------------------------------------------------------------------------------

To apply the aforementioned formalism to the Hawking radiation, one recognises 
the universe in the CGHS model as the total isolated system $U$, whereas the 
gravitational degrees of freedom as $E$, and the system in concern $S$ is the 
radiation field. It has been shown in \cite{Demers1996,Kiefer2001} that the 
reduced state of the radiation field can indeed be \emph{thermal} in certain 
\emph{decoherence} schemes, while in the collapsing case discussed in 
\cref{sec:hawrad-1+1dila,sec:1+1ddlt}, the radiation field remains pure.

In this section, the trace distance between the pure and thermal descriptions 
will be evaluated. Due to technical difficulties, the distance has not been able 
to be derived exactly. Instead, a lower and an upper bound have been set to the 
distance by \emph{Fuchs-van de Graaf inequality} in \cref{eq:ineq-fvdg}, where 
only the calculable \emph{Fidelity} (see \cref{sec:fidelity}) is needed.

Note that the wave functional of the Hawking radiation field 
\cref{eq:squeezed-wave-functional} is, roughly speaking, the superposition of 
quantum-mechanical wave functions per Fourier mode,
\begin{equation}
\sfun{\chi_b}{g} \sim \sum_p \rfun{\chi_b^\rbr{p}}{g_p}
= \sum_p \Braket{g_p | \chi_b^\rbr{p}},
\label{eq:wave-function-decomposed}
\end{equation}
where $g_p \coloneqq \rfun{g}{p}$ is the Fourier transform of the field $g$ 
evaluated at $p$. On the other hand, the thermal state of the free field can 
also be sloppily written as the product of the quantum-mechanical density 
operators per mode,
\begin{equation}
{\rho}_\text{th} \sim \bigotimes_p {\rho}_\text{th}^\rbr{p}.
\end{equation}
By \cref{eq:fidelity-pure-mixed}, the fidelity of $\sfun{\chi_b}{g}$ and 
$\rho_\text{th}$ can be reduced to the product of the fidelity per mode, because
\begin{equation}
\Braket{\chi_b | {\rho}_\text{th} | \chi_b} \sim \prod_p 
\Braket{\chi_b^\rbr{p} | {\rho}_\text{th}^\rbr{p} | \chi_b^\rbr{p}}.
\label{eq:expectation-decomposed}
\end{equation}

Since $\Ket{\chi_b^\rbr{p}}$ is just a quantum-mechanical general Gaussian 
state (see \cref{sec:single-harosc}), the result in \cref{eq:fidelity-gg-th} 
can be adapted by substituting
\begin{equation}
\Omega = \vbr{p},\qquad \omega = p\,\coth\frac{\pp p}{2\lambda}\quad
\text{and}\quad T = T_\text{HD} \equiv \frac{\lambda}{2\pp},
\end{equation}
yielding the \emph{fidelity per mode}
\begin{equation}
F^\rbr{p} = \frac{\sqrt{u-1}}{\sqrt[4]{u^2+u+1}},\qquad
u \coloneqq \ee^q \equiv \ee^{\vbr{p}/T_\text{HD}},
\label{eq:fidelity-pt}
\end{equation}
so that the trace distance per Fourier mode can be evaluated, see 
\cref{fig:trace-dist-per-mode}.

\begin{figure}
\begin{center}
\input{./graphics/graph_trace_dist_mode_bounds}
\end{center}
\caption[Possible value of the trace distance per Fourier mode]{Possible value 
of the trace distance between mode wave functions in 
\cref{eq:wave-function-decomposed} and the corresponding thermal density 
operators. One sees that the difference becomes exponentially small with respect 
to $p/T_\text{HD}$, which suggests it would be difficult to distinguish the pure 
and thermal descriptions by detecting the high-energy modes in the Hawking 
radiation, confirming the results in \cref{sec:corr_Fourier}.
\label{fig:trace-dist-per-mode}}
%, evaluated in terms of lower and upper bound by fidelity
\end{figure}

To approach the trace distance for the wave functional and the total thermal 
state, one needs to deal with the product with continuous index in 
\cref{eq:expectation-decomposed}. A popular way to go to the `continuous limit' 
in \emph{summation} is
\begin{equation}
\sum_p \rfun{g}{p} \to \frac{1}{2\pp\Lambda} \int \dd p\,\rfun{g}{p},
\end{equation}
where $\Lambda$ has the same dimension as $p$ in order to fix the dimension; in 
the box-normalisation scheme, for example, the corresponding $\Lambda$ would be 
proportional to the volume of the box $V$. A similar method may be used to 
normalise the product, namely
\begin{equation}
\prod_p \rfun{f}{p} = \cfun{\exp}{\sum_p \ln\rfun{f}{p}} \to 
\cfun{\exp}{\frac{1}{2\pp\Lambda}\int \dd p\,\ln\rfun{f}{p}}.
\end{equation}
Note that the wave functional \cref{eq:squeezed-wave-functional}, which has 
always been dealt with, can also be seen as being normalised by the method.
One thus derives
\begin{equation}
F = \cfun{\exp}{\frac{2}{2\pp\Lambda}\int_0^{+\infty}\dd p\,\ln F^\rbr{p}}
= \rfun{\exp}{-\frac{\pp}{9} \frac{T_\text{HD}}{\Lambda}},
\end{equation}
which is shown in \cref{fig:trace-dist-total}.

\begin{figure}
\begin{center}
\input{./graphics/graph_trace_dist_total_bounds}
\end{center}
\caption[Possible value of the trace distance between pure and thermal 
descriptions]{Possible value of the trace distance between the pure and thermal 
descriptions of the Hawking radiation field. One sees that the difference 
between the two cases is positively (negatively) correlated with the Hawking 
temperature temperature (mass of black hole). 
\label{fig:trace-dist-total}}
\end{figure}

%------------------------------------------------------------------------------
\section{Discussion}
\label{sec:dist-discussion}
%------------------------------------------------------------------------------
Strictly speaking, the formalism in \cref{sec:arbitrary-canonical} is \emph{not}
applicable to the radiation field considered here, because they were proved for 
\emph{finite dimensional systems}, while the cases in this work are all 
\emph{infinite-dimensional}. However, since a regularised result can be 
obtained, it is believable that a mathematically rigorous approach exists, which 
will justify the result, similar to the renormalisation procedure in quantum 
field theory.

Though the asymptotic behaviour of the trace distance with respect to Hawking 
temperature has clearly been revealed by setting bounds to it, it is still 
appealing to calculate its \emph{exact value}. One possible approach is by 
using the results in \cite{Joos1985}, where the eigenfunctions and eigenvalues 
of a general Gaussian density operator have been explicitly solved.

The evaluation of the trace distance was motivated by \citeauthor{Hsu2009} in 
\cite{Hsu2009}, where operational meaning of the distance was also discussed in 
terms of toy models, which is unfortunately not applicable to the field system. 
In future works, it is expected that the distance between the pure and thermal 
states will be understood in terms of \emph{experiments} or \emph{observations}, 
where the essence of physical science lies.


%%% Local Variables: 
%%% mode: latex
%%% TeX-master: "../mythesis"
%%% End: 
