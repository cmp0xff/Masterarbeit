%==============================================================================
\chapter{Correlator of Field Strength}
\label{chap:corr_field}
%==============================================================================

Having revealed the discrepancy in the kinematic description of Hawking 
radiation, namely as either a pure or a thermal and mixed state, the author 
will show the difference in a physically relevant form. Recall that the 
particle-number expectations of the pure and mixed states are the same, 
i.e.\ the diagonal elements of the density operators in the particle-number 
basis are the same. Hence one seeks operators revealing the off-diagonal 
elements of the density operator.

%------------------------------------------------------------------------------
\section{Correlator and Fluctuation of Fourier Modes}
\label{sec:corr_Fourier}
%------------------------------------------------------------------------------

A natural candidate to reveal the off-diagonal elements of the density 
operators is the correlator. Bearing in mind the correlator of generalised 
Gaussian wave functions (see \cref{eq:multi-har-cor}), the correlator of the 
wave functional \cref{eq:squeezed-wave-functional} can be read off as
\begin{align}
\abr{\rfun{g^\dagger}{p_1}\rfun{g}{p_2}}_{\chi_b} &=
\frac{1}{2}\rbr{\abr{\rfun{g_\Re}{p_1}\rfun{g_\Re}{p_2}}_{\chi_b}+\abr{
\rfun{g_\Im}{p_1}\rfun{g_\Im}{p_2}}_{\chi_b}}
= \frac{1}{2}\frac{\tanh\frac{\pp p_1}{2\lambda}}{p_1} \rfun{\delta}{p_1 - 
p_2} \nonumber \\
&\propto \frac{1}{8T_\text{HD}}\frac{\rfun{\tanh}{q/4}}{q/4} = 
\frac{1}{8T_\text{HD}}\rbr{1 + \rfun{\Omicron}{q}}
\label{eq:correlator-pure}
\end{align}
by substituting
\begin{equation}
\rfun{g}{p} = 2^{-1/2} \rbr{\rfun{g_\Re}{p} + \ii \rfun{g_\Im}{p}},
\end{equation}
where $2^{-1/2}\rfun{g_\Re}{p}$ and $2^{-1/2}\rfun{g_\Im}{p}$ are the 
real and imaginary part of $\rfun{g}{p}$, respectively. In the last line of 
\cref{eq:correlator-pure}, the delta function is ignored, and the dimensionless 
parameter
\begin{equation}
	q \coloneqq p_1/T_\text{HD} \equiv 2\pp p_1/\lambda
\end{equation}
has been used. By the same argument, one also solves the correlator
\begin{equation}
\abr{\rfun{g^\dagger}{p_1}\rfun{g}{p_2}}_\text{vac} =
\frac{1}{2\vbr{p_1}}\rfun{\delta}{p_1-p_2} \propto
\frac{1}{2 T_\text{HD}} q^{-1}
\label{eq:correlator-vacuum}
\end{equation}
for the vacuum wave functional $\cfun{\exp}{-\int_{-\infty}^{+\infty}\dd p \, 
\vbr{p}\vbr{\rfun{g}{p}}^2}$.
%\begin{equation}
%\label{eq:vacuum-wave-functional-g}
%\end{equation}
For the thermal state, on the other hand, the correlator follows from
\cref{eq:correlator-multiple-thermal}, so that
\begin{align}
\abr{\rfun{g^\dagger}{p_1}\rfun{g}{p_2}}_\text{th} &=
\frac{\coth\frac{\pp p_1}{\lambda}}{2 p_1} \rfun{\delta}{p_1-p_2}
\nonumber \\
&\propto \frac{1}{4T_\text{HD}} \frac{\rfun{\coth}{q/2}}{q/2}
= \frac{1}{4T_\text{HD}}\rbr{\rbr{\frac{q}{2}}^{-2} + \frac{1}{3} + 
\rfun{\Omicron}{q}}.
\label{eq:correlator-thermal}
\end{align}

One sees immediately that 
\cref{eq:correlator-pure,eq:correlator-vacuum,eq:correlator-thermal} vanish 
identically for $p_1 \neq p_2$, which is due to the absence of interaction in 
the scalar field. The remaining diagonal terms have the meaning of 
\emph{fluctuation} in field strength, because $\abr{g} \equiv 0$, so that
\begin{equation}
	\abr{\rbr{\Delta g}^2} = \abr{g^2} - \abr{g}^2 \equiv \abr{g^2}.
\end{equation}
Ignoring the $\rfun{\delta}{0}$ divergence, the fluctuations are plotted in
\cref{fig:fluc-Fourier}.

\begin{figure}
\begin{center}
\begin{tikzpicture}
\begin{loglogaxis}[
xlabel = {$p \cdot T_\text{HD}^{-1}$},
extra x ticks = {3e-2, 3e-1, 3e0},
xticklabel={
        \pgfkeys{/pgf/fpu=true}
        \pgfmathparse{exp(\tick)}%
        \pgfmathprintnumber[fixed relative, precision=3]{\pgfmathresult}
        \pgfkeys{/pgf/fpu=false}
      },
ylabel = {$\abr{\vbr{\rfun{g}{p}}^2}\cdot T_\text{HD}$},
legend entries={vacuum,pure,thermal}]
\addplot[smooth]
table[x index = 0, y index = 1] {./graphics/mesh_fluc_fourier.dat};
\addplot[smooth, dashed]
table[x index = 0, y index = 2] {./graphics/mesh_fluc_fourier.dat};
\addplot[smooth, dashdotted]
table[x index = 0, y index = 3] {./graphics/mesh_fluc_fourier.dat};
\end{loglogaxis}
\end{tikzpicture}
\end{center}
\caption[Fluctuations of the Fourier modes]{Fluctuations of the Fourier modes 
of the field, plotted in log--log scales. The critical energy scale is the
Hawking temperature, below which the various fluctuations depart. The 
discrepancy between the pure and thermal descriptions is most significant for 
low-energy modes, while for high energies the fluctuations of them and 
the vacuum state are practically the same. Moreover, compared with vacuum, the 
low-energy fluctuation of thermal case is enhanced, so it diverges faster than 
that of vacuum; for the pure description, however, the fluctuation is 
suppressed, such that it converges to a constant of order unity and does not 
diverge any more. \label{fig:fluc-Fourier}}
\end{figure}

%------------------------------------------------------------------------------
\section{Discussion}
\label{sec:corr-discussion}
%------------------------------------------------------------------------------

It has been noticed that the fluctuation discussed above has been extensively 
used in cosmology, see e.g.\ \cite{Glenz2009}. The fluctuation in the 
radiation field, however, looks different from that in cosmology and thus 
remains yet to be explained.

Moreover, the most natural candidate to reveal the off-diagonal elements is the 
correlator in \emph{real space}, which is just the Fourier transform of the 
correlator calculated above. Unfortunately, the transformation has yet to be 
made for the thermal state due to an infra-red divergence.

%%% Local Variables: 
%%% mode: latex
%%% TeX-master: "../mythesis"
%%% End: 

