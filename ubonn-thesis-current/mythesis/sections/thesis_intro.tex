%==============================================================================
\chapter{Introduction}
\label{sec:intro}
%==============================================================================

\epigraph{Our mistake is not that we take our theories too seriously, but that 
we do not take them seriously enough.}{Steven Weinberg 
\cite{Weinberg1993first}}

Although widely accepted, Hawking radiation remains a field of heated 
discussion and active research, especially in the interpretation and 
extrapolation of it. Being one of the first conclusions of the original 
calculation, Hawking temperature was deduced from a \emph{pure state} of the 
quantum field in the background space-time of a collapsing body, whereas a 
temperature in statistical physics is usually derived from a statistical 
ensemble in equilibrium, described by a \emph{thermal} and \emph{mixed state}. 
This discrepancy itself has since long been largely ignored, albeit related 
issues have always been in spotlight, for instance the \emph{information loss 
problem} , the \emph{origin of black hole entropy} \cite{Mann2015,Harlow2016}, 
etc. A detailed investigation of the pure and thermal descriptions would help 
understanding the aforementioned questions by laying them on a more solid 
foundation.

In this work, which is motivated by \cite{Kiefer2001,Hsu2009}, the focus is to 
reveal and quantify the difference between the two cases mentioned above. 
\Cref{chap:hawrad} is a review of Hawking radiation in the 
$\rbr{3+1}$-dimensional Einstein gravitation which is basically along 
\citeauthor{Hawking1976}'s original line, as well as that in 
$\rbr{1+1}$-dimensional dilaton gravity, also known as the  
Callan--Giddings--Harvey--Strominger (CGHS) model, in which the quantum field 
theory in curved space-time not only can be \emph{derived} from the full theory 
of quantum gravity, but also be solved exactly. Then in \cref{chap:corr_field}, 
motivated by the coincidence of the particle-number expectations, which are the 
diagonal elements of the density operators, the author computes the correlator 
of field strength, so as to reveal the difference in the off-diagonal elements. 
In \cref{chap:quantifing}, inspired by a new foundation of statistical physics, 
which is based on exact results in quantum information theory, the author 
quantifies the discrepancy between the two cases by calculating the trace 
distance between them, and the results are shown to be in accordance with the 
quantum-informational foundation, as well as those in \cref{chap:corr_field}. 
\Cref{chap:summary} is summary and outlook.

In the appendices, \cref{sec:1+1ddlt} explains more details about the CGHS 
model, \cref{chap:harosc} collects some useful results in quantised simple 
harmonic oscillator, and \cref{chap:distappd} introduces trace distance and 
fidelity.

Throughout the text, the \emph{natural units} will be used unless specified, 
where the speed of light in vacuum $\lc$, the reduced Planck constant $\phs$ and 
Boltzmann constant $\bk$ are all set to unity, while the Newton constant $\nG$ 
is kept.




%%% Local Variables: 
%%% mode: latex
%%% TeX-master: "../mythesis"
%%% End: 
