%==============================================================================
\chapter{Harmonic Oscillators}
\label{chap:harosc}
%==============================================================================

%------------------------------------------------------------------------------
\section{Single Harmonic Oscillator}
\label{sec:single-harosc}
%------------------------------------------------------------------------------

A single harmonic oscillator is described by the Hamiltonian
\begin{equation}
H = \frac{1}{2}\rbr{\pi^2 + \Omega^2f^2},
\label{eq:hamiltonian-simple-harosc}
\end{equation}
where $\rbr{\pi, f}$ are conjugate momentum and position, respectively. Here
a unit system similar to the case in field theory is adapted. Canonical
quantisation uses the annihilation and creation operators
\begin{equation}
a^- \coloneqq \frac{1}{\sqrt{2}}\rbr{\sqrt{\Omega}f + 
\frac{\ii}{\sqrt{\Omega}}\pi}, \qquad
a^+ \coloneqq \frac{1}{\sqrt{2}}\rbr{\sqrt{\Omega}f
-\frac{\ii}{\sqrt{\Omega}}\pi} \equiv \rbr{a^-}^\dagger;
\end{equation}
inverse expressions read
\begin{equation}
f = \frac{a^++a^-}{\sqrt{2\Omega}},\qquad
\pi = \ii\sqrt{\frac{\Omega}{2}}\rbr{a^+-a^-}.
\end{equation}
The ground state and the normalised $n$th excitation are defined by
\begin{equation}
a^-\Ket{0} \coloneqq 0,\qquad
\Ket{n} \coloneqq \frac{1}{\sqrt{n!}}\rbr{a^+}^n\Ket{0},\qquad
n \in \BbbZ_+,
\end{equation}
the wave functions of which are
\begin{equation}
\Braket{f | n} = \frac{1}{\sqrt{2^n n!}} 
\ee^{-\Omega f^2/2} \rfun{H_n}{\sqrt{\Omega}f}, \qquad
n \in \BbbZ,
\end{equation}
where $\rfun{H_n}{x}$ is the $n$th Hermite polynomial. The normalising measure
\begin{equation}
\rfun{\dd \mu}{f} \coloneqq \sqrt{\frac{\Omega}{\pp}}\,\dd f
\end{equation}
is used throughout, so that the completeness relation holds,
\begin{align}
\int \rfun{\dd \mu}{f}\,\Braket{\alpha | f}\Braket{f | \beta} \equiv
\Braket{\alpha | \beta}.
\end{align}

In the present work, a general Gaussian state $\Ket{\omega}$ has also been 
considered, the wave function of which reads
\begin{equation}
\Braket{f | \omega} = \rbr{\frac{\Re\omega}{\Omega}}^{1/4}
\rfun{\exp}{-\frac{\omega}{2}f^2},\qquad \Re \omega > 0
\label{eq:general-single-omega}
\end{equation}
Evaluating the expectation of $f^2$ yields
\begin{equation}
\abr{f^2}_\omega \coloneqq \Braket{\omega | f^2 | \omega}
= \rbr{2\Re \omega}^{-1}.
\label{eq:correlator-1}
\end{equation}
$\Ket{\omega}$ can also be expressed in terms of energy eigenstates,
\begin{align}
\Braket{n | \omega} &= \int \rfun{\dd \mu}{f}\, \Braket{n | f} 
\Braket{f | \omega} \nonumber \\
&= \rbr{\frac{\Omega \Re\omega}{\pp^2}}^{1/4} \frac{1}{\sqrt{2^n n!}}
\int_{-\infty}^{+\infty} \dd f\,
\cfun{\exp}{-\frac{1}{2}\rbr{\Omega+\omega}f^2}\rfun{H_n}{\sqrt{\Omega}f}
\nonumber \\
&= \begin{cases} \rbr{\Omega  \Re(\omega )}^{\frac{1}{4}}
\dfrac{2^{\frac{1}{2}-m} \sqrt{(2 m)!} }{m!}
\dfrac{(\Omega -\omega )^m}{(\omega +\Omega )^{m+\frac{1}{2}}}
& n = 2m, \\
0 & n = 2m+1,
\end{cases}
\label{eq:n|omega}
\end{align}
thanks to \cite{Babusci2012}.

A thermal state of the oscillator at temperature $T$ can be described by the
density operator
\begin{equation}
\rho \coloneqq \frac{1}{Z} \cfun{\exp}{-\frac{H}{T}} = \frac{1}{Z} 
\sum_{n=0}^{+\infty} \cfun{\exp}{-\frac{\Omega}{T}\rbr{n+\frac{1}{2}}} 
\Ket{n}\Bra{n},
% \nonumber \\
%&= Z^{-1} \sum_{n=0}^{+\infty} 
%\frac{\ee^{-\frac{\Omega}{T}\rbr{n+\frac{1}{2}}}}{n!}
%\rbr{a^+}^n\Ket{0}\Bra{0}\rbr{a^-}^n,
\label{eq:single-thermal-density-mat}
\end{equation}
where the partition function is
\begin{equation}
Z \coloneqq \tr{\ee^{-H/T}} = \frac{1}{2} \csch\frac{\Omega}{2 T}.
\end{equation}
One also obtains
\begin{align}
\abr{f^2}_\rho &= \frac{1}{2\Omega}\frac{1}{Z}\sum_{n=0}^{+\infty}
\cfun{\exp}{-\frac{\Omega}{T}\rbr{n+\frac{1}{2}}}
\Braket{n | \rbr{a^++a^-}^2 | n} \nonumber \\
%&= \frac{1}{2\Omega}\frac{1}{Z}\sum_{n=0}^{+\infty}
%\ee^{-\frac{\Omega}{T}\rbr{n+\frac{1}{2}}}
%\Braket{n | \rbr{\rbr{a^+}^2+a^+a^-+a^-a^++\rbr{a^-}^2} | n} \nonumber \\
&= \frac{1}{2\Omega}\frac{1}{Z}\sum_{n=0}^{+\infty}
\cfun{\exp}{-\frac{\Omega}{T}\rbr{n+\frac{1}{2}}}
\Braket{n | \rbr{2a^+a^-+1} | n} %\nonumber \\
%&= \frac{\coth\frac{\Omega}{2 T}}{2\Omega}
= \frac{1}{4T} \frac{\coth\frac{\Omega}{2 T}}{\frac{\Omega}{2 T}}.
\label{eq:sho-th-cor-1}
\end{align}
%\Cref{eq:single-thermal-density-mat} expressed in terms of position basis can
%also be calculated
%\begin{align}
%\rho &= \frac{1}{Z}
%\sum_{n=0}^{+\infty} \ee^{-\frac{\Omega}{T}\rbr{n+\frac{1}{2}}}
%\int\rfun{\dd\mu}{f_1}\,\rfun{\dd\mu}{f_2}\,
%\Ket{f_1}\Braket{f_1 | n}\Braket{n | f_2}\Bra{f_2}
%\nonumber \\
%&= \frac{1}{Z} \int\rfun{\dd\mu}{f_1}\,\rfun{\dd\mu}{f_2}\,\Ket{f_1}
%\rbr{\sum_{n=0}^{+\infty}\ee^{-\frac{\Omega}{T}\rbr{n+\frac{1}{2}}}
%\Braket{f_1 | n}\Braket{n | f_2}}
%\Bra{f_2};
%\label{eq:thermal-position-basis-1}
%\end{align}
%The summation in \cref{eq:thermal-position-basis-1} can be closed
%\begin{equation}
%\rfun{\dd\mu}{f_1}\,\rfun{\dd\mu}{f_2}\,
%\sum_{n=0}^{+\infty}\Braket{f_1 | n}\Braket{n | f_2}
%= \dd f_1\,\dd f_2 \rfun{\delta}{f_1 - f_2},
%\end{equation}
%due to orthonormality and completeness of the bases.

%------------------------------------------------------------------------------
\section{Multiple Harmonic Oscillators}
%------------------------------------------------------------------------------

Multiple harmonic oscillators are described by the Hamiltonian
\begin{equation}
H = \sum_i H_i,\qquad H_i = \frac{1}{2}\rbr{\pi_i^2 + \Omega^2f_i^2},
\end{equation}
where $\rbr{\pi_i, f_i}$ are conjugate momentum and position of the $i$th
oscillator.

When the general Gaussian state
\begin{equation}
\Braket{\cbr{f} | \sbr{\omega}} = \rbr{\det\frac{\omega}{\Omega}}^{1/4} 
\rfun{\exp}{-\frac{1}{2}f_i \omega_{ij} f_j}
\label{eq:general-ij-omega}
\end{equation}
is considered, where $\omega_{ij}$ is a real symmetric positive-definite
matrix, the two-point correlator becomes
\begin{equation}
\Braket{\sbr{\omega} | f_i f_j | \sbr{\omega}} = \sbr{\rbr{2\omega}^{-1}}_{ij},
\label{eq:multi-har-cor}
\end{equation}
which is a generalisation of \cref{eq:correlator-1}.

A thermal state at temperature $T$ can be described by the density operator
\begin{equation}
\rho = \bigotimes_i \rho_i,\qquad \rho_i =
\frac{1}{Z} \rfun{\exp}{-\frac{H_i}{T}}.
\label{eq:multi-thermal-density-mat}
\end{equation}
Computing the two-point correlator of the state, one finds
\begin{equation}
\abr{f_if_j}_\rho = \frac{\coth\frac{\Omega_i}{2T}}{2\Omega_i}
\delta_{ij}
= \frac{1}{4T} \frac{\coth\frac{\Omega_i}{2T}}{\frac{\Omega_i}{2T}} 
\delta_{ij}
\label{eq:correlator-multiple-thermal}
\end{equation}
from \cref{eq:sho-th-cor-1}.

%%% Local Variables: 
%%% mode: latex
%%% TeX-master: "../mythesis"
%%% End: 
