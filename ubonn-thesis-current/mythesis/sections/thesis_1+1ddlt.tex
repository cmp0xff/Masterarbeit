%==============================================================================
\chapter{$\rbr{1+1}$-dimensional Dilaton Gravity}
\label{sec:1+1ddlt}
%==============================================================================

Since quantum field theories in $\rbr{3+1}$-dimensional Einstein gravitation 
are difficult to solve, one may turn to alternative solvable gravity models to 
get some hints for the physics in reality. The $\rbr{1+1}$-dimensional dilaton 
gravity, or the Callan--Giddings--Harvey--Strominger (CGHS) model, is such a 
candidate, which is used in the thesis and has been extensively studied in the 
literature, see for instance \cite{Callan1992,Demers1996,Ashtekar2011}. In this 
chapter a very brief review of the model is provided, mainly based on 
\cite{Demers1996}.

The action for the model, with $N$ massless scalar fields minimally coupled, 
reads
\begin{equation}
S \coloneqq \int \dd^2 x \sqrt{-\ol{g}}\cbr{\frac{\ee^{-2\ol{\phi}}}{\nG}
\sbr{\ol{R}+4\rbr{\nabla \ol{\phi}}^2 + 4\lambda^2}
-\frac{1}{2}\sum_{i=1}^N\rbr{\nabla f_i}^2},
\label{eq:action-CGHS-dilaton}
\end{equation}
where $f_i$'s are the neutral scalar matter fields, $\ol{\phi}$ is the dilaton 
field, $\nG$ the dimensionless Newton constant, and $\lambda > 0$ the 
cosmological constant. The dilaton field is essential in two dimensions, because 
there is only one independent component in the Riemann curvature tensor, hence a 
pure $\rbr{1+1}$-dimensional Einstein theory shall be trivial. Transforming 
with $\phi = \ee^{-2\ol{\phi}}$ and $g_{\alpha\beta} = \ee^{-2\ol{\phi}} 
\ol{g}_{\alpha\beta}$ eliminates the kinetic term for the dilaton, yielding
\begin{equation}
S = \int \dd x\,\dd t\sqrt{-g}\cbr{\frac{1}{\nG}\sbr{R\phi + 4\lambda^2}
-\frac{1}{2}\rbr{\nabla f}^2},
\label{eq:action-CGHS-dilaton-eli}
\end{equation}
where only one matter field is considered for simplicity, and an 
ADM-like\footnote{Arnowitt--Deser--Misner, see \cite{Arnowitt2008}.}
parametrisation of the metric
\begin{equation}
\dd s^2 = \ee^{2\rho}\sbr{-\sigma^2\,\dd t^2 + \rbr{\dd x + \xi\,\dd t}^2}
\label{eq:ADM-para}
\end{equation}
is assumed, in which $\rbr{\sigma, \xi}$ are the lapse and shift functions.
The action in \cref{eq:action-CGHS-dilaton-eli} has a classical solution 
describing a collapsing null-matter shell, which resembles the solution of 
a spherically collapsing body in the $\rbr{3+1}$-dimensional Einstein case. 
The corresponding conformal diagram is plotted in \cref{fig:dil-col-bod}.

To apply the canonical quantisation scheme, the action in
\cref{eq:action-CGHS-dilaton-eli} is to be recast in the Hamiltonian formalism.
However, the Legendre transformation of the field momenta proves to be 
singular. This means that the momenta, as functions of the positions and 
velocities, cannot be inverted to express the corresponding velocities as 
functions of momenta and positions, so that the standard algorithm to obtain 
the Hamiltonian
\begin{equation}
H \coloneqq \sbr{\frpa{L}{\dot{X}_i}\dot{X}_i - L}_{\dot{X}
= \rfun{\dot{X}}{P,X}}
\end{equation}
does not apply, where $\rbr{X, \dot{X}, P}$ are the positions, velocities and 
momenta, respectively.

The systems, of which the Legendre transformation is singular, are called 
\emph{constrained systems}. Other examples include a relativistic point 
particle in the covariant form, Yang--Mills theories and string theories. For 
such systems, the usual quantisation scheme and the Schrödinger equation do 
not apply directly. Instead, one has to identify the constraints in the system 
and apply Dirac's quantisation rules \cite{dirac1964lectures,Gitman1990}. The 
result is that the quantum wave functional describing the CGHS model is 
\emph{constrained} by
\begin{align}
0 = \mscrH_\parallel\sfun{\Psi}{\rho,\phi,f} &\coloneqq
\rbr{\frde{\rho}{x}\frdva{}{\rho} - \frde{}{x}\frdva{}{\rho} 
+\frde{\phi}{x}\frdva{}{\phi} + \frde{f}{x}\frdva{}{f}}\sfun{\Psi}{\rho,\phi,f},
\label{eq:hpara}\\
0 = \mscrH_\perp\sfun{\Psi}{\rho,\phi,f} &\coloneqq
\rbr{\frac{\nG}{2}\frdva{^2}{\rho\,\dva\phi} - \frac{1}{2}\frdva{^2}{f^2} 
+\frac{1}{2\nG}V_G + V_M}\sfun{\Psi}{\rho,\phi,f},
\label{eq:hperp}
\end{align}
where
\begin{equation}
V_G \coloneqq 4\rbr{\frde{^2\phi}{x^2} - \frde{\phi}{x}\frde{\rho}{x} - 
2\lambda^2 \ee^{2\rho}},\qquad
V_M \coloneqq\frac{1}{2}\rbr{\frde{f}{x}}^2.
\end{equation}
\Cref{eq:hpara,eq:hperp} are the \emph{Wheeler--DeWitt equations} for the CGHS 
model, which play the role of the usual Schrödinger equation for the whole 
system.

In the next step, a semi-classical approximation (see also \cite[sec.\ 
5.4]{Kiefer2012}) of the Born--Oppenheimer type is applied to $\Psi$ by 
expanding the exponent as
\begin{equation}
\sfun{\Psi}{\rho,\phi,f} = \ee^{\ii\rbr{\nG^{-1}S_0 + S_1 + \nG S_2 + \ldots}}.
\end{equation}
Inserting this expression into \cref{eq:hpara,eq:hperp}, one finds that at 
order $\nG^{0}$, variables can be separated by setting
\begin{equation}
\ee^{\ii S_1} \coloneqq \sfun{D^{-1}}{\rho,\phi}\sfun{\chi}{\rho,\phi,f}.
\end{equation}
Inserting the leading and next-to-leading order terms into 
\cref{eq:hpara,eq:hperp} yields
\begin{equation}
\ii \rbr{\frpa{\rho}{t}\frdva{\chi}{\rho} + \frpa{\phi}{t}\frdva{\chi}{\phi}} = 
\frac{1}{2}\cbr{-\frdva{^2}{f^2}+\rbr{\frpa{f}{x}}^2} \chi,
\end{equation}
integrating of which gives the functional Schrödinger equation for the matter 
field \eqref{eq:functional-Sch-chi}.


% Order $\nG^{-1}$: Hamilton--Jacobi for pure gravity


%%% Local Variables: 
%%% mode: latex
%%% TeX-master: "../mythesis"
%%% End: 
