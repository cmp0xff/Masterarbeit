%==============================================================================
\chapter{Summary and Outlook}
\label{chap:summary}
%==============================================================================

In this work, the discrepancy of the pure-state and thermal descriptions of 
Hawking radiation field has been revealed, based on the CGHS gravity model. It 
has been shown that in fluctuations of Fourier modes, the difference is 
significant for low-energy excitations while vanishing for high-energy ones, 
which fits the result that the Hawking-radiation particles are mostly 
low-energetic, compared with the Hawking temperature. Moreover, motivated by the 
new foundation of statistical physics, the discrepancy between the two 
descriptions has been quantified and proved to be large for a low-mass 
black hole, which is expected to show more traces of quantum gravity.

The CGHS model used throughout is only one of the solvable alternative gravity 
models. One may further attempt other popular choices, for example, the 
\emph{Bañados--Teitelboim--Zanelli model} \cite{Banados1992} in $\rbr{2+1}$ 
dimensions, and compare the relevant results.

Equipped with the results in the thesis, one expects deeper understandings 
about the nature of the black-hole entropy and the final fate of the black-hole 
evaporation, which have been in the spotlight since the advent of Hawking 
radiation.

In future works, the decoherence formalism will also be taken into account, in 
which a thermalised reduced state of the radiation field can be obtained 
explicitly.