% $Header: /Users/joseph/Documents/LaTeX/beamer/solutions/conference-talks/conference-ornate-20min.en.tex,v 90e850259b8b 2007/01/28 20:48:30 tantau $

\documentclass{beamer}

\usefonttheme{professionalfonts}%  don't change fonts inside beamer
% This file is a solution template for:

% - Talk at a conference/colloquium.
% - Talk length is about 20min.
% - Style is ornate.



% Copyright 2004 by Till Tantau <tantau@users.sourceforge.net>.
%
% In principle, this file can be redistributed and/or modified under
% the terms of the GNU Public License, version 2.
%
% However, this file is supposed to be a template to be modified
% for your own needs. For this reason, if you use this file as a
% template and not specifically distribute it as part of a another
% package/program, I grant the extra permission to freely copy and
% modify this file as you see fit and even to delete this copyright
% notice. 


\mode<presentation>
{
  % \usetheme{Warsaw}
  % or ...

  \setbeamercovered{transparent}
  % or whatever (possibly just delete it)
}


%\usepackage[english]{babel}
% or whatever

%\usepackage[latin1]{inputenc}
% or whatever
\usepackage{polyglossia}
\setmainlanguage{english}
\usepackage{fontspec}
\usepackage{xeCJK}
\usepackage{unicode-math}
	\setmathfont{Latin Modern Math} % default
	%\setmathfont[range=\mathalpha]{Asana Math}
	\setmathfont{Asana Math}[range={\mathbin}] %\mathord
	\setmathfont{STIX Math}[range={"02609}] % ☉
	\setmathfont{XITS Math}[range={"1D4B6-"1D4CF}] % Script, Latin, lowercase
	\setmathfont{Latin Modern Math}[range={"1D608-"1D63B}, sans-style=italic]
	\setmathfont{Latin Modern Math}[range={
		"00391-"003A9,
		"003B1-"003F5, 
		"1D6A8-"1D6E1},	% Bold Greek
		sans-style=upright]
		
	%\setmathfont{⟨font name⟩}[range=⟨unicode range⟩,⟨font features⟩]
\usepackage{siunitx}
% ':=' as \coloneqq
%\usepackage{mathtools}
\usepackage{empheq} % numcases


%\usepackage{times}
%\usepackage[T1]{fontenc}
% Or whatever. Note that the encoding and the font should match. If T1
% does not look nice, try deleting the line with the fontenc.


\title%[Short Paper Title] % (optional, use only with long paper titles)
{Hawking Radiation}

\subtitle{A Comparison of Pure-state and Thermal Description}

\author[Wang] % (optional, use only with lots of authors)
{Yi-Fan Wang 王一帆}
%\inst{1} \and %S.~Another\inst{2}}
% - Give the names in the same order as the appear in the paper.
% - Use the \inst{?} command only if the authors have different
%   affiliation.

\institute[Uni zu Köln] % (optional, but mostly needed)
{
  %\inst{1}%
  Institut für Theoretische Physik \\
  Universität zu Köln}
  %\and
  %\inst{2}%
  %Department of Theoretical Philosophy\\
  %University of Elsewhere}
% - Use the \inst command only if there are several affiliations.
% - Keep it simple, no one is interested in your street address.

\date%[ASGRC WS1617] % (optional, should be abbreviation of conference name)
{Masterkolloquium}
% - Either use conference name or its abbreviation.
% - Not really informative to the audience, more for people (including
%   yourself) who are reading the slides online

\subject{General Relativity and Cosmology}
% This is only inserted into the PDF information catalog. Can be left
% out. 



% If you have a file called "university-logo-filename.xxx", where xxx
% is a graphic format that can be processed by latex or pdflatex,
% resp., then you can add a logo as follows:

\pgfdeclareimage[height=1.0cm]{university-logo}{./bcgs_logo_vektorgrafikNeu}
\logo{\pgfuseimage{university-logo}}



% Delete this, if you do not want the table of contents to pop up at
% the beginning of each subsection:
\AtBeginSection[]
{
  \begin{frame}<beamer>{Outline}
    \tableofcontents[currentsection,currentsubsection]
  \end{frame}
}


% If you wish to uncover everything in a step-wise fashion, uncomment
% the following command: 

%\beamerdefaultoverlayspecification{<+->}

\usepackage[citestyle=alphabetic,doi=false,isbn=false,url=false,
defernumbers=true]%
	{biblatex}
\addbibresource{kolloquium.bib}

\usepackage{tikz}
% \usepackage{tikz-3dplot}
% \usetikzlibrary{positioning,shapes,arrows}
\usetikzlibrary{decorations.pathmorphing}
\usetikzlibrary{calc}
% \usetikzlibrary{decorations.markings}

\usepackage{pgfplots}
\pgfplotsset{compat=1.13}
\usepgfplotslibrary{fillbetween}

\usepackage{cleveref}

\usepackage{braket}

% Mathematical constants
\newcommand{\ii}{{\Bbbi}}
\newcommand{\ee}{{\Bbbe}}
\newcommand{\pp}{{\Bbbpi}}

% Bracket-like
\newcommand{\rbr}[1]{{\left(#1\right)}}
\newcommand{\sbr}[1]{{\left[#1\right]}}
\newcommand{\cbr}[1]{{\left\{#1\right\}}}
\newcommand{\abr}[1]{{\left<#1\right>}}
\newcommand{\vbr}[1]{{\left|#1\right|}}
\newcommand{\fat}[2]{{\left.#1\right|_{#2}}}
% Functions; note the space between the name and the bracket!
\newcommand{\rfun}[2]{{#1}\mathopen{}\left(#2\right)\mathclose{}}
\newcommand{\sfun}[2]{{#1}\mathopen{}\left[#2\right]\mathclose{}}
\newcommand{\cfun}[2]{{#1}\mathopen{}\left\{#2\right\}\mathclose{}}
\newcommand{\afun}[2]{{#1}\mathopen{}\left<#2\right>\mathclose{}}
\newcommand{\vfun}[2]{{#1}\mathopen{}\left|#2\right|\mathclose{}}
% Differentials
\newcommand{\Dif}{\BbbD}
\newcommand{\Diff}{\,\BbbD}
\newcommand{\dd}{\Bbbd}
\newcommand{\ddf}{\,\Bbbd}
\newcommand{\dva}{\mupdelta} % no better way?!
\newcommand{\dvar}{\,\mupdelta}
% Fraction-like
\newcommand{\frde}[2]{{\frac{\dd{#1}}{\dd{#2}}}}
\newcommand{\frDe}[2]{{\frac{\Dif{#1}}{\Dif{#2}}}}
\newcommand{\frpa}[2]{{\frac{\partial{#1}}{\partial{#2}}}}
% Equal marks
\newcommand{\eeq}{{\overset{!}{=}}}
\newcommand{\lls}{{\overset{!}{<}}}
\newcommand{\ggt}{{\overset{!}{>}}}
\newcommand{\lle}{{\overset{!}{\le}}}
\newcommand{\gge}{{\overset{!}{\ge}}}
% overline-like marks
\newcommand{\ol}[1]{{\overline{{#1}}}}
\newcommand{\ul}[1]{{\underline{{#1}}}}
\newcommand{\tld}[1]{{\widetilde{{#1}}}}
\newcommand{\ora}[1]{{\overrightarrow{#1}}}
\newcommand{\ola}[1]{{\overleftarrow{#1}}}
\newcommand{\td}[1]{{\widetilde{#1}}}
\newcommand{\what}[1]{{\widehat{#1}}}
%\newcommand{\prm}{{\symbol{"2032}}}

% Math operators
% Why does \DeclareMathOperator not work?
\DeclareMathOperator{\sgn}{sgn}
\DeclareMathOperator{\grad}{grad}
\DeclareMathOperator{\curl}{curl}
\DeclareMathOperator{\rot}{rot}
\DeclareMathOperator{\opdiv}{div}
\DeclareMathOperator{\opdeg}{deg}

\DeclareMathOperator{\sech}{sech}
\DeclareMathOperator{\csch}{csch}

\DeclareMathOperator{\diag}{diag}
\DeclareMathOperator{\tr}{tr}

\DeclareMathOperator{\ad}{ad}

\DeclareMathOperator{\expi}{expi}

% Group and Algebras
\newcommand{\SO}{\mathsf{SO}\,}
\newcommand{\SU}{\mathsf{SU}\,}
\newcommand{\so}{\mathfrak{so}\,}
\newcommand{\su}{\mathfrak{su}\,}

% Physical constants
\newcommand{\nG}{\mitsansG} % Newton's constant


\begin{document}

\begin{frame}
  \titlepage
\end{frame}

\begin{frame}{Outline}
  \tableofcontents
  % You might wish to add the option [pausesections]
\end{frame}


% Structuring a talk is a difficult task and the following structure
% may not be suitable. Here are some rules that apply for this
% solution: 

% - Exactly two or three sections (other than the summary).
% - At *most* three subsections per section.
% - Talk about 30s to 2min per frame. So there should be between about
%   15 and 30 frames, all told.

% - A conference audience is likely to know very little of what you
%   are going to talk about. So *simplify*!
% - In a 20min talk, getting the main ideas across is hard
%   enough. Leave out details, even if it means being less precise than
%   you think necessary.
% - If you omit details that are vital to the proof/implementation,
%   just say so once. Everybody will be happy with that.

%\section{Motivation}

%\subsection{The Basic Problem That We Studied}

%\begin{frame}{Make Titles Informative. Use Uppercase Letters.}{Subtitles are 
%optional.}
  % - A title should summarize the slide in an understandable fashion
  %   for anyone how does not follow everything on the slide itself.

%  \begin{itemize}
%  \item
%    Use \texttt{itemize} a lot.
%  \item
%    Use very short sentences or short phrases.
%  \end{itemize}
%\end{frame}

%\begin{frame}{Make Titles Informative.}

%  You can create overlays\dots
%  \begin{itemize}
%  \item using the \texttt{pause} command:
%    \begin{itemize}
%%    \item
%      First item.
%      \pause
%    \item    
%      Second item.
%    \end{itemize}
%  \item
%    using overlay specifications:
%    \begin{itemize}
%    \item<3->
%      First item.
%    \item<4->
%      Second item.
%    \end{itemize}
%  \item
%    using the general \texttt{uncover} command:
%    \begin{itemize}
%      \uncover<5->{\item
%        First item.}
%      \uncover<6->{\item
%        Second item.}
%    \end{itemize}
%  \end{itemize}
%\end{frame}

\section{Hawking Radiation}

\begin{frame}{Hawking Radiation}{Background space-time: sperically collapsing body}
\begin{center}
\begin{tikzpicture}%[scale=2]
\pgfmathsetmacro\myunit{3} 
\pgfmathsetmacro\sc{1.41421 35623 73095 04880 16887 24209 69807 85696 71875}
\pgfmathsetmacro\grs{0.6180339887498949}
\pgfmathsetmacro\grb{1.6180339887498949}
	\draw [decorate, decoration=zigzag] (0,0)
		-- ++ (+  0: \sc * \grs * \myunit)
			node [above] {$i^+$}
			node [pos = .5, above] {bh sing.}
			coordinate (i+);
	\draw [dotted] (0,0)
			\pgfextra{\pgfmathparse{(\grs/\sc+\sc)*\myunit}}
		-- ++ (- 90: \pgfmathresult)
			node [below] {$i^-$}
			node [pos = .31, right] {collapsing}
			node [pos = .62, right] {body}
			coordinate (i-);
	\draw (i+)
		\pgfextra{\pgfmathparse{(1 - \grs/2)*\myunit}}
		-- ++ (- 45: \pgfmathresult)
			node [right] {$i^0$}
			node [pos = .5, above right, sloped] {$\mscrI^+$}
			coordinate (i0)
		-- (i-)
			node [pos = .5, below right, sloped] {$\mscrI^-$};
	\draw [dashed] (i+)
		-- ++ (-135: 2 * \grs * \myunit)
			node [pos = .4, above left, sloped] {$\mscrh^+$};
	\draw [thick, out = 67.5, in = -80, thick, ->] (i-)
			to (\grs * \myunit, 0);
\end{tikzpicture}

\end{center}
\end{frame}

\begin{frame}{Hawking Radiation}{Result and interpretation}

\begin{itemize}
\item An early-time \alert{vacuum} on $\mscrI^-$ in collapsing background
\begin{equation}
\rfun{\what{a}}{p}\Ket{h} \coloneqq 0\quad\Rightarrow\quad
\abr{\rfun{\what{n}_a}{p}}_h \equiv 0
\end{equation}
evolves to a late-time state on $\mscrI^+\cup\mscrh^+$ \alert{with particles}
\cite{HAWKING1974}
\begin{equation}
\abr{\rfun{\what{n}_b}{\omega}}_h \approx
\Gamma_\omega\rbr{\ee^{2\pp\omega/\kappa}-1}^{-1}.
\label{eq:hawking-distri}
\end{equation}
\item Comparing \cref{eq:hawking-distri} with the Bose--Einstein distribution
\begin{equation}
\abr{\rfun{\what{n}}{\omega}}_\text{BE} = \rbr{\ee^{\omega/T} - 1}^{-1}, 
\end{equation}
one may conclude that \cref{eq:hawking-distri} describes a grey-body
radiation with the \alert{Hawking temperature} \cite{Hawking1975},
\begin{equation}
T_\text{H} \coloneqq \kappa/2\pp.
\end{equation}

\end{itemize}

\end{frame}


\begin{frame}{Hawking Radiation}{Tension in the interpretation}
\begin{itemize}
\item The state $\Ket{h}$ or its density operator is \alert{pure},
\begin{equation}
\what{\rho}_h = \Ket{h}\Bra{h},
\end{equation}
whilst the Bose--Einstein state of equilibrium bosonic gas
\begin{equation}
\what{\rho}_\text{BE} = {Z}^{-1} \ee^{-\what{H}/T}
\sim Z^{-1} \sum_E \ee^{-E/T} \Ket{E}\Bra{E}
\end{equation}
is \alert{thermal} and \alert{mixed}.
\item How different are they? \cite{Kiefer2001,Hsu2009}
\end{itemize}
\end{frame}

\section{Model of Gravitation}

\begin{frame}{$\rbr{1+1}$-dimensional Dilaton Gravity Model}{Classical theory
\only<1>{1}\only<2>{2}/2 \cite{Callan1992,Demers1996,Ashtekar2011}}

\only<1>{
\begin{itemize}
\item The action of the dilaton gravity model reads
\begin{align}
S &= \int \dd^2 x\,\sqrt{-\ol{g}}\,\cbr{\frac{\ee^{-2\ol{\phi}}}{\nG}
\sbr{\ol{R}+4\rbr{\nabla \ol{\phi}}^2 + 4\lambda^2}
-\frac{1}{2}\rbr{\nabla f}^2} \nonumber \\
&= \int \dd^2 x\,\sqrt{-g}\,\cbr{\frac{1}{\nG}\sbr{R\phi + 4\lambda^2}
-\frac{1}{2}\rbr{\nabla f}^2},
\label{eq:action-CGHS-dilaton-eli}
\end{align}
where in \cref{eq:action-CGHS-dilaton-eli}, the \alert{kinetic term} of the 
dilaton field $\phi$ is eliminated by substituting $\phi = \ee^{-2\ol{\phi}}$ 
and $g_{\alpha\beta} = \ee^{-2\ol{\phi}} \ol{g}_{\alpha\beta}$.
\item Has a solution which \alert{resembles} the collapsing body in 
$\rbr{3+1}$d Einstein gravitation
\end{itemize}}

\only<2>{
\begin{center}
\begin{tikzpicture}%[scale=2]
\pgfmathsetmacro\myunit{3}
\pgfmathsetmacro\grs{0.6180339887498949}
\pgfmathsetmacro\grb{1.6180339887498949}
	\draw (0,0)
			node [left] {$i^0_\text{L}$}
		-- ++ (- 45: \grb * \myunit)
			node [pos = .50, below left, sloped] {$\mscrI^-_\text{L}$}
			node [pos = .10, right = .3*\myunit cm] {dilaton}
			node [pos = .35, right = .3*\myunit cm] {vacuum}
			node [pos = .60, right = .3*\myunit cm] {region}
			node [below] {$i^-$}
		-- ++ (+ 45: \myunit)
			%node [above = .25*\myunit cm] {region}
			coordinate (one)
		-- ++ (+ 45: \grs * \myunit)
			node [pos = .75, left = .15*\myunit cm] {region}
			node [right] {$i^0_\text{R}$}
			coordinate (r-inf)
		-- ++ (+135: \myunit)
			node [pos = .70, left = .35*\myunit cm] {black}
			node [pos = .45, left = .25*\myunit cm] {hole}
 			node [pos = .50, above right, sloped] {$\mscrI^+_\text{R}$}
			node [above] {$i^+_\text{R}$}
			coordinate (r-i+)
			;

	\draw (0,0)
		-- ++ (+ 45: \myunit)
			node [pos = .4, right = .25*\myunit cm] {linear}
			node [above] {$i^+_\text{L}$}
			coordinate (l-i+);
			%node [below = .55*\myunit cm] {linear}
			%node [below = .75*\myunit cm] {dilaton}
			%node [below = .95*\myunit cm] {vacuum};

	\draw [dashed] (r-i+)
		-- ++ (-135: \grb * \myunit)
			coordinate (mlambda);
		
	\draw [thick, ->] (one)
		-- ++ (+135: 0.5 * \myunit)
			coordinate (bb);
	\draw [thick, ->] (bb)
		-- ++ (+135: 0.5 * \myunit)
			coordinate (cross);
	\draw [thick, ->] (cross) -- (l-i+);

	\draw [decorate, decoration=zigzag] (l-i+)
		-- (r-i+)
		node [pos = .5, above] {bh sing.};

	\draw plot [smooth, dash dot] coordinates {(0,0) (cross) (r-inf)};
\end{tikzpicture}
\end{center}}

\end{frame}


\begin{frame}{$\rbr{1+1}$-dimensional Dilaton Gravity Model}{Quantum theory
\only<1>{1}\only<2>{2}/2 \cite{Demers1996}}

\begin{itemize}
\only<1>{
\item Can formally be canonically quantised as a constraint system

\item Apply a \alert{Born--Oppenheimer}-type approximation to 
$\sfun{\Psi}{g,\phi,f}$ \cite[§~5.4]{Kiefer2012}

\begin{itemize}
\item Separate `\emph{Heavy, slow}' gravity and `\emph{light, fast}' matter
\item Apply WKB approximation to the gravity part
\item Assume disentanglement $\sfun{\Psi}{g,\phi,f} = \sfun{D}{g, \phi}
\sfun{\chi}{g, \phi, f}$
\end{itemize}

\item Insert the ansatz $\sfun{\Psi}{g,\phi,f} = \ee^{\ii\rbr{\nG^{-1}S_0 + 
S_1 + \nG S_2 + \ldots}}$

\begin{itemize}
\item Order $\nG^{-1}$: Hamilton--Jacobi equation for pure gravity
\item Order $\nG^{0}$: functional \alert{Schrödinger equation for matter}
\begin{align}
\ii \frpa{\chi}{t} &= \what{H}_\text{m} \chi,
\label{eq:functional-sch} \\
\what{H}_\text{m} &\equiv \frac{1}{2}\int_{0}^{+\infty} \dd k\,
\rbr{-\frac{\mbfdelta^2}{\mbfdelta \rfun{f^2}{k}} + k^2 \rfun{f^2}{k}}.
\end{align}
\end{itemize}}

\only<2>{
\item At early time, the \alert{ground-state} solution to 
\cref{eq:functional-sch} is 
%(eq.\ (48))
\begin{equation}
\sfun{\chi_0}{f} \propto
	\cfun{\exp}{-\frac{1}{2}\int_{0}^{+\infty}\dd k\, k \, \rfun{f}{k}^2},
\end{equation}
while at late time it evolves to %(eq.\ (57) in \cite{Demers1996})
\begin{equation}
\sfun{\chi_b}{g} \propto 
\cfun{\exp}{-\int_{-\infty}^{+\infty}\dd p\, p \,\rfun{\coth}{\frac{\pp 
p}{2\lambda}} \vbr{\rfun{g}{p}}^2},
\label{eq:squeezed-wave-functional}
\end{equation}
where $\rfun{f}{k}$ and $\rfun{g}{p}$ are the Fourier transform of the matter 
field at early and late time, respectively.
%\begin{equation}
%\rfun{g}{k} = \int_{-\infty}^{+\infty}\dd v\,
%\frac{\ee^{-\ii k v}}{\sqrt{2\pp}} \rfun{f}{v},
%\end{equation}

\item At late time, \alert{particle-number expectations} are
\begin{equation}
\abr{\rfun{\what{n}_b}{p}}_{\chi_b} = \rbr{\ee^{2\pp\vbr{p}/\lambda}-1}^{-1},
\end{equation}
leading to a Hawking-like \alert{black-body temperature}
\begin{empheq}[box=\fbox]{equation}
T_\text{HD} \coloneqq \lambda/2\pp.
\label{eq:hawking-dilaton}
\end{empheq}}

\end{itemize}



%\item the wave functional of which 
%has a semi-classical expansion (eq.\ (10) in \cite{Demers1996})
%\begin{equation}
%\sfun{\Psi}{\rho,\phi,f} = \ee^{\ii\rbr{\nG^{-1}S_0 + S_1 + \nG S_2 + \ldots}}
%\end{equation}

\end{frame}

\section{Correlator of Field Strength}

%\begin{frame}{Correlation of the Field Strength}{Introduction}
%\begin{itemize}
%\item Measures the res
%\end{itemize}


%\end{frame}

\begin{frame}{Correlation of Fourier Modes}{The discrepancy}

%\begin{empheq}[left=\empheqlbrace]{align}
%&\frac{\pp}{4\lambda} q^{-1}, &\quad \text{vacuum};
%\label{eq:spectrum-gs} \\
%&\frac{1}{8T_\text{HD}}\frac{\rfun{\tanh}{q/4}}{q/4};
%\label{eq:spectrum-pure} \\
%&\frac{1}{4T_\text{HD}} \frac{\rfun{\coth}{q/2}}{q/2},
%&\quad \text{thermal},
%\label{eq:spectrum-thermal}
%\end{empheq}
The Fourier-mode correlators of different states can be calculated,
\begin{equation}
\abr{\rfun{\what{g}^\dagger}{p_1}\rfun{\what{g}}{p_2}} =
\frac{1}{T_\text{HD}} \rfun{\delta}{p_1 - p_2} \cdot
\begin{cases}
\frac{1}{2} \frac{1}{q},
&\quad \text{vacuum};\\
\frac{1}{8}\frac{\tanh\frac{q}{4}}{\frac{q}{4}},
&\quad \chi_b; \\
\frac{1}{4} \frac{\coth\frac{q}{2}}{\frac{q}{2}},
&\quad \rfun{\what{\rho}_\text{BE}}{T_\text{HD}},
\end{cases}
\end{equation}
where $q \coloneqq p_1/T_\text{HD}$.

Note that
\begin{equation}
\abr{\rbr{\mbfDelta\what{g}}^2} = \abr{\what{g}^2} - \abr{\what{g}}^2 = 
\abr{\what{g}^2} .
\end{equation}
\end{frame}

\begin{frame}{Correlation of Fourier Modes}{Fluctuation of the Fourier 
modes: diagram in log-log scale}

\begin{center}
\begin{tikzpicture}
\begin{loglogaxis}[
xlabel = {$p \cdot T_\text{HD}^{-1}$},
extra x ticks = {3e-2, 3e-1, 3e0},
xticklabel={
        \pgfkeys{/pgf/fpu=true}
        \pgfmathparse{exp(\tick)}%
        \pgfmathprintnumber[fixed relative, precision=3]{\pgfmathresult}
        \pgfkeys{/pgf/fpu=false}
      },
ylabel = {$\abr{\vbr{\rfun{\what{g}}{p}}^2}\cdot T_\text{HD}$},
legend entries={vacuum,pure,thermal}]
\addplot+[smooth, mark = none]
table[x index = 0, y index = 1] {./graphics/mesh_fluc_fourier.dat};
\addplot+[smooth, mark = none]
table[x index = 0, y index = 2] {./graphics/mesh_fluc_fourier.dat};
\addplot+[smooth, mark = none]
table[x index = 0, y index = 3] {./graphics/mesh_fluc_fourier.dat};
\end{loglogaxis}
\end{tikzpicture}
\end{center}

\end{frame}

\begin{frame}{Correlation of Fourier Modes}{Fluctuation of the Fourier 
modes: interpretation}

\begin{itemize}
	\item Vacuum fluctuation
	\begin{equation}
	\abr{\vbr{\what{g}}^2}_\text{vac} \sim \rfun{\Omicron}{q^{-1}}
	\end{equation}

	\item A black hole does \alert{not} alter the \alert{high}-energy
	processes.
	\begin{equation}
	\abr{\vbr{\what{g}}^2}_{\chi_b} \approx
	\abr{\vbr{\what{g}}^2}_\text{th} \approx
	\abr{\vbr{\what{g}}^2}_\text{vac} \sim \rfun{\Omicron}{q^{-1}},
	\quad \vbr{p} \gg T_\text{HD}
	\end{equation}

	\item A black hole \alert{suppresses} low-energy fluctuation of the
	\alert{pure} state, while \alert{enhancing} that of the
	\alert{thermal} state.
	\begin{equation}
	\rfun{\Omicron}{1} \sim \abr{\vbr{\what{g}}^2}_{\chi_b} \ll
	\abr{\vbr{\what{g}}^2}_\text{vac} \ll
	\abr{\vbr{\what{g}}^2}_\text{th} \sim \rfun{\Omicron}{q^{-2}},
	\quad \vbr{p} \ll T_\text{HD}
	\end{equation}
	\item Critical scale $\vbr{p} \sim T_\text{HD}$
\end{itemize}

\end{frame}

\section{Distance between Density Operators}

\begin{frame}{Another New Foundation of Statistical Physics 
\cite{Popescu2006}}{Specific and easy version of the construction}

\begin{itemize}
\item Total isolated system $U$ with \alert{energy} constraint $\abr{\what{H}_U} 
\coloneqq E_U$, divided into a \alert{(sub)system} $S$ and an environment $E$

\item Hilbert spaces $\mscrH_R \supseteq \mscrH_U = \mscrH_S \otimes \mscrH_E$;
$\what{1}_R$ identity on $\mscrH_R$, dimension $d_R \coloneqq \dim \mscrH_R < 
+\infty$

\item Equiprobable / maximal-ignorant state of $U$
\begin{equation}
\what{\mscrE}_R \coloneqq d_R^{-1} \what{1}_R \in \mscrH_R
\end{equation}

\item Hamiltonians $\what{H}_U = \what{H}_S + \what{H}_E + \what{H}_\text{int}$

\item \alert{Canonical state} of $S$ with energy 
constraint \cite[§~28]{Landau1980}
\begin{equation}
\what{\Omega}_S^\rbr{\text{E}} \coloneqq \tr_E \what{\mscrE}_R 
\propto\rfun{\exp}{- 
\what{H}_S/T_\text{th}}
\end{equation}

\item \alert{Theorem}: $\forall \Ket{\phi} \in \mscrH_R$, the reduced state of 
$S$
\begin{empheq}[box=\fbox]{equation}
\tr_E \Ket{\phi}\Bra{\phi} \eqqcolon \rfun{\what{\rho}_S}{\phi} \approx 
\what{\Omega}_S^\rbr{\text{E}}.
\end{empheq}

\end{itemize}

\end{frame}


\begin{frame}{Another New Foundation of Statistical Physics 
\cite{Popescu2006}}{Generic and exact version of the construction}

\begin{itemize}
\item \alert{Arbitrary} constraint $R$; study the \alert{trace 
distance} $T$ between $\rfun{\what{\rho}_S}{\phi}$ and $\what{\Omega}_S$

\item Lemma: \alert{average distance} is small w.r.t.\ $d_S/d_E^\text{eff}$
\begin{equation}
\abr{\rfun{T}{\rfun{\what{\rho}_S}{\phi}, \what{\Omega}_S}} \le 
\tfrac{1}{2}\sqrt{d_S/d_E^\text{eff}}
\end{equation}

\item Theorem: \alert{probability of large deviation} is exponentially 
small w.r.t.\ the distance; an easy version
\begin{equation}
\frac{\sfun{V}{\cbr{\Ket{\phi} \,\Big |\,
\rfun{T}{\rfun{\what{\rho}_S}{\phi}, \what{\Omega}_S} \geq d_R^{-\frac{1}{3}}}}}
{\sfun{V}{\cbr{\Ket{\phi} }}} \leq
4\rfun{\exp}{- \frac{2d_R^{\frac{1}{3}}}{9\pp^3} }
\end{equation}

\item Effective dimension of $E$: setting
$\what{\Omega}_E = \tr_S \what{\mscrE}_R$,
\begin{equation}
d_U/d_S \equiv d_E \ge d_E^\text{eff}
\coloneqq \rbr{\tr \what{\Omega}_E^2}^{-1} \ge d_R/d_S.
\end{equation}

\end{itemize}

\end{frame}

\begin{frame}{Trace Distance \cite[ch.~9]{Wilde2009}}{Definitions}
\begin{itemize}
\item Trace distance between Hermitian operators $\what{M}$ and $\what{N}$
\begin{equation}
\rfun{T}{\what{M}, \what{N}} \coloneqq \frac{1}{2} 
\tr \sqrt{\rbr{\what{M}-\what{N}}^\dagger\rbr{\what{M}-\what{N}}}
\label{eq:def-trace-dist}
\end{equation}
\begin{itemize}
\item Positivity, homogeneity, triangle ineq., isometric inv.
\end{itemize}
\item For \alert{density operators} $\what{\rho}$ and $\what{\sigma}$,
\begin{itemize}
\item $0 \le \rfun{T}{\what{\rho}, \what{\sigma}} \le 1.$
\item Controlled by \alert{fidelity} in \cite{Fuchs1999}
\begin{equation}
1 - \rfun{F}{\what{\rho},\what{\sigma}} \le \rfun{T}{\what{\rho},\what{\sigma}}
\le \sqrt{1 - \rfun{F^2}{\what{\rho},\what{\sigma}}},
\label{eq:ineq-fvdg}
\end{equation}
where we only need
\begin{equation}
\rfun{F}{\Ket{\alpha},\what{\sigma}} \coloneqq \sqrt{\Braket{\alpha | 
\what{\sigma} | \alpha}}.
\end{equation}
\item In our application, $T$ is difficult while $F$ can be obtained.
\end{itemize}
\end{itemize}
\end{frame}

\begin{frame}{Trace Distance \cite[ch.~9]{Wilde2009}}{Interpretation}
\begin{itemize}
\item Maximal probability-difference obtainable
\begin{equation}
\rfun{T}{\what{\rho}, \what{\sigma}} = \max_{0 \le \what{\Lambda} \le \what{1}}
\cfun{\tr}{\what{\Lambda}\rbr{\what{\rho}-\what{\sigma}}},
\end{equation}
where all eigenvalues of $\what{\Lambda}$ are in the range $\sbr{0,1}$
\item E.g.\ $\what{\Lambda} \coloneqq \Ket{\alpha}\Bra{\alpha}$, $\Ket{\alpha}$
eigenstate of $\what{\Alpha}$ with eigenvalue $\alpha$
\begin{itemize}
\item $\cfun{\tr}{\what{\Lambda}\what{\rho}}$: the probability of getting
$\alpha$ in measuring $\what{\Alpha}$
\item $\cfun{\tr}{\what{\Lambda}\rbr{\what{\rho}-\what{\sigma}}}$: the
difference of the probability above
\item $\rfun{T}{\what{\rho}, \what{\sigma}}$: the maximal value of the 
difference above
\end{itemize}
\end{itemize}

\end{frame}




%\begin{frame}{Distances between the Density Operators}{Fidelity: introduction}
%\begin{itemize}
%\item Fidelity \cite[ch.~6]{Petz2008}
%\begin{align}
%\rfun{F}{\Ket{\alpha},\Ket{\beta}} &= \vbr{\Braket{\alpha | \beta }} \\
%\rfun{F}{\Ket{\alpha},\what{\sigma}} &= \sqrt{\Braket{\alpha | \what{\sigma} | 
%\alpha}} \\
%\rfun{F}{\what{\rho},\what{\sigma}} &= \tr\sqrt{\what{\rho}^\frac{1}{2} 
%\what{\sigma} \what{\rho}^\frac{1}{2}}
%\end{align}



%\end{itemize}

%\end{frame}

\begin{frame}{Distances between the Density Operators}{\only<1-2>{Single-mode 
trace distances}\only<3-4>{All-mode trace distance}\only<2,4>{, bounds set by 
fidelity and \cref{eq:ineq-fvdg}}}

\only<1>{
\begin{itemize}
%\item Consider a general Gaussian state
%\begin{equation}
%\Braket{g | \omega} 
%\propto \ee^{-\frac{\omega}{2}g^2}
%\end{equation}
%of a harmonic oscillator $\what{H} = \frac{1}{2}\rbr{\what{\pi}^2
%+ \Omega^2 \what{g}^2}$.

%\item $\rfun{F}{\Ket{\omega}, \rfun{\what{\rho}}{T}}$ can be exactly calculated,
%with $\rfun{\what{\rho}}{T}$ being thermal at arbitrary temperature $T$

%\item Taking $\Omega = p$, $\omega = p\,\coth\frac{\pp p}{2\lambda}$ 
%and $T = T_\text{HD}$ (%
%\crefrange{eq:squeezed-wave-functional}{eq:hawking-dilaton}),
\item $\sfun{\chi}{f} \sim S$, $\sfun{\Psi}{g, \phi, f} \sim U$

\item The pure wave functional can sloppily be decomposed $\sfun{\chi_b}{g} 
\sim \sum_p \rfun{\chi_b^\rbr{p}}{g_p} \equiv \sum_p \Braket{g_p | 
\chi_b^{\rbr{p}}}$, where $g_p \coloneqq \rfun{g}{p}$

\item So does the thermal density operator $\rfun{\what{\rho}_\text{th}}{T} \sim
\bigotimes_p \rfun{\what{\rho}_\text{th}^\rbr{p}}{T}$

\item Fidelity in \cref{eq:ineq-fvdg} can be factorised as well
\begin{equation}
F \equiv \Braket{\chi_b | \what{\rho}_\text{th} | \chi_b}^\frac{1}{2} \sim
\prod_p \Braket{\chi_b^{\rbr{p}} | \what{\rho}_\text{th}^{\rbr{p}}
| \chi_b^{\rbr{p}}}^\frac{1}{2} \eqqcolon \prod_p F^{\rbr{p}};
\end{equation}
\item $F^{\rbr{p}}$ can be computed \alert{in order to find bounds of $T$}
\begin{equation}
\rfun{F^{\rbr{p}}}{\Ket{p},\rfun{\what{\rho}_\text{th}^\rbr{p}}{T_\text{HD}}}
= \frac{\sqrt{u-1}}{\sqrt[4]{u^2+u+1}},\quad
u \coloneqq \ee^q \equiv \ee^{\vbr{p}/T_\text{HD}}.
\label{eq:fidelity-pt}
\end{equation}
\end{itemize}}

\only<2>{
\begin{center}
\begin{tikzpicture}
\begin{semilogyaxis}[
xlabel = {$p \cdot T_\text{HD}^{-1}$},
ylabel = {$\rfun{T}{\Ket{p},
	\rfun{\what{\rho}_\text{th}^\rbr{p}}{T_\text{HD}}}$},
legend entries={lower bound,upper bound,trace distance},
legend pos = south west]
\addplot+[name path = lower, smooth, mark = none]
	table [x index = 0, y index = 1] 
		{./graphics/mesh_trace_dist_mode_bounds.dat};
\addplot+[name path = upper, smooth, mark = none]
	table [x index = 0, y index = 2] 
		{./graphics/mesh_trace_dist_mode_bounds.dat};
\addplot+[brown!50]
	fill between [of = lower and upper];
\end{semilogyaxis}

\end{tikzpicture}
\end{center}
%Difference goes exponentially small w.r.t.\ $p$ or $\rbr{\text{wave 
%length}}^{-1}$
}

\only<3>{
\begin{itemize}
%\item Inserting \cref{eq:fidelity-pt} into \cref{eq:ineq-fvdg} and expanding 
%$u$ at $\infty$ yields
%\begin{equation}
%\frac{3}{4} \ee^{-\frac{2 \pp  p}{\lambda }} \le 1-F \le T \le \sqrt{1-F^2} \le
%\sqrt{\frac{3}{2}} \ee^{-\frac{\pp  p}{\lambda }}
%\end{equation}

\item `Go to the continuum limit': $\Lambda$ dimension regulator
\begin{equation}
\sum_p \rfun{g}{p} \to \frac{1}{2\pp\Lambda} \int \dd p\,\rfun{g}{p},
\end{equation}

\item Analogously, to regularising a positive product
\begin{equation}
\prod_p \rfun{f}{p} \equiv \cfun{\exp}{\sum_p 
\ln\rfun{f}{p}} \to \cfun{\exp}{\frac{1}{2\pp\Lambda}
\int \dd p\,\ln\rfun{f}{p}}
\end{equation}

\item Regularised $F$ can be calculated \alert{in order to set bounds of $T$}
\begin{equation}
\rfun{F}{\chi_b, \what{\rho}_\text{th}} = 
\cfun{\exp}{\frac{2}{2\pp\Lambda}\int_0^{+\infty}\dd p\,\ln F^{\rbr{p}}} =
\rfun{\exp}{-\frac{\pp}{9} \frac{T_\text{HD}}{\Lambda}}
\end{equation}
\end{itemize}}

\only<4>{
\begin{center}
\begin{tikzpicture}
\begin{loglogaxis}[
xlabel = {$T_\text{HD} \cdot \Lambda^{-1}$},
extra x ticks = {3e-1, 3e0},
xticklabel={
        \pgfkeys{/pgf/fpu=true}
        \pgfmathparse{exp(\tick)}%
        \pgfmathprintnumber[fixed relative, precision=3]{\pgfmathresult}
        \pgfkeys{/pgf/fpu=false}
      },
ylabel = {$\rfun{T}{\chi_b, \rfun{\what{\rho}_\text{th}}{T_\text{HD}}}$},
extra y ticks = {3e-1, 3e-2},
yticklabel={
        \pgfkeys{/pgf/fpu=true}
        \pgfmathparse{exp(\tick)}%
        \pgfmathprintnumber[fixed relative, precision=3]{\pgfmathresult}
        \pgfkeys{/pgf/fpu=false}
      },
legend entries={lower bound,upper bound,trace distance},
legend pos = south east]
\addplot+[name path = lower, smooth, mark = none]
	table [x index = 0, y index = 1] 
		{./graphics/mesh_trace_dist_total_bounds.dat};
\addplot+[name path = upper, smooth, mark = none]
	table [x index = 0, y index = 2] 
		{./graphics/mesh_trace_dist_total_bounds.dat};
\addplot+[brown!50]
	fill between [of = lower and upper];
\end{loglogaxis}

\end{tikzpicture}
\end{center}
%Difference goes exponentially small w.r.t.\ $T_\text{HD}$ or $M^{-1}$
}

\end{frame}



\section*{Summary}

\begin{frame}{Summary}

  % Keep the summary *very short*.
\begin{itemize}
\item
Compared the \alert{pure} and the \alert{thermal} descriptions of the
radiation within the solvable dilaton gravity model
\item
Fourier-mode fluctuation: that of the thermal state \alert{diverges} faster
than the vacuum case does at low energy, while the pure-state fluc.\ remains
\alert{finite}; at high energies they \alert{converge}.
\item
Trace distance: goes exponentially \alert{small} with black hole 
\alert{temperature} going to zero.
\end{itemize}
  
  % The following outlook is optional.
  \vskip0pt plus.5fill
  \begin{itemize}
  \item
    Outlook
    \begin{itemize}
	\item Understand the dilaton gravity model
	\item Understand the Fourier-mode fluctuation
	\item Evaluate the real space correlator
	\item Understand practical meaning of trace distance
	\item Understand the regulator in total trace distance
	\item Evaluate the exact trace distance
	\item Include the gravitational wave functional
    \end{itemize}
  \end{itemize}
\end{frame}



% All of the following is optional and typically not needed. 
\appendix
\section<presentation>*{\appendixname}
\subsection<presentation>*{For Further Reading}

\begin{frame}[allowframebreaks]
  \frametitle<presentation>{For Further Reading}
    
%  \begin{thebibliography}{10}
    
  \beamertemplatebookbibitems
  % Start with overview books.
\printbibliography[type=book]

%  \bibitem{Author1990}
%    A.~Author.
%    \newblock {\em Handbook of Everything}.
%    \newblock Some Press, 1990.
 
    
  \beamertemplatearticlebibitems
  % Followed by interesting articles. Keep the list short. 
\printbibliography[nottype=book]

%  \bibitem{Someone2000}
%    S.~Someone.
%    \newblock On this and that.
%    \newblock {\em Journal of This and That}, 2(1):50--100,
%    2000.
%  \end{thebibliography}
\end{frame}



\end{document}


