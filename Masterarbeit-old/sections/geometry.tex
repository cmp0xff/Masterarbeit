\section{Aspects of space-time geometry}
\label{sec:geometry}

\subsection{Aspects of the fundamentals}

%R_{\mu\nu}-\frac{1}{2}R g_{\mu\nu} + \Lambda g_{\mu\nu} = 8\pp\nG T_{\mu\nu}.

Cauchy surface

Globally hyperbolic

Surface gravity

\cite{carroll2004spacetime}

\cite{griffiths2009exact}

\subsection{Maximally symmetric space-times}

\begin{nameddef}{Unit $2$-sphere}
\begin{equation}
\dif\Omega^2 = \dif\theta^2 + \sin^2\theta\diff\phi^2
\end{equation}
is the metric of unit $2$-sphere $S^2$.
\end{nameddef}

\begin{nameddef}{Spherically symmetric space-times}
All spherically symmetric space-times in this work possess coordinates
in which the metric takes the form
\begin{equation}
\dif s^2 = -\rfun{f}{r}\diff t^2 + \rfun{f}{r}^{-1}\dif r^2
+ r^2 \diff \Omega^2,
\label{eq:generic-s-metric}
\end{equation}
which is manifest in its $\rfun{\SO}{3}$ symmetry.
\end{nameddef}

\begin{nameddef}{Spherical coordinates of maximally symmetric space-times}
\begin{equation}
\rfun{f}{r} = 1 - \frac{\nG\Lambda}{3}r^2.
\end{equation}
\end{nameddef}

\begin{equation}
\frac{t+r}{R_0}=\tan\frac{\eta+\chi}{2},\qquad
\frac{t-r}{R_0}=\tan\frac{\eta-\chi}{2}.
\end{equation}
\begin{nameddef}{Conformal coordinates}
\begin{equation}
\dif s^2 = \frac{R_0^2}{\rbr{\cos\eta+\cos\chi}^2}
\rbr{-\dif\eta^2+\dif\chi^2+\sin^2\chi\diff\Omega^2}
\end{equation}
\end{nameddef}

\subsection{Schwarzschild space-time}

\subsubsection{Coordinates}

\begin{nameddef}{Schwarzschild coordinates}
\label{nd:sch-coord}

%\cite{schwarzschild1916}

The metric of Schwarzschild space-time
is given by \cref{eq:generic-s-metric} with
\begin{equation}
\rfun{f}{r} = \rbr{1-\frac{\rSch}{r}},
\label{eq:co-scsc}
\end{equation}
%\begin{equation}
%\dif s^2 = -\rbr{1-\frac{\rSch}{r}}\diff t^2+\rbr{1-\frac{\rSch}{r}}^{-1}
%\diff r^2 + r^2\diff\Omega^2,
%\end{equation}
where $\rSch = 2\nG M$ is the Schwarzschild radius.
%The ranges of three of the coordinates are
%$-\infty < t < +\infty$, $0 < \theta < \pp$; $0 < \phi < 2\pp$; the range of
%$r$ is $\rbr{0, \rSch} \cup \rbr{\rSch, +\infty}$, making the coordinates covering
%two disjoint patches of the space-time.
\end{nameddef}


To exploit the Weyl-flatness of $\mathcal{M}/S^2$, one may transform the
holonomic co-frame by
\begin{equation}
\dif r^* = \rbr{1-\frac{\rSch}{r}}^{-1}\diff r,
\label{eq:trsf-dtosc}
\end{equation}
which can be integrated to
\begin{equation}
\ee^{r_*/\rSch-1} =
\rbr{r/\rSch-1}\ee^{r/\rSch-1},
\label{eq:trsf-gtosc}
\end{equation}
generating the transformations
\begin{equation*}
r_* = r+\rSch\rfun{\ln}{\frac{r}{\rSch}-1}\quad\text{and}\quad
r = \rSch\rbr{1+\rfun{W}{\ee^{r_*/\rSch-1}}},
%\label{eq:trsf-scto}
\end{equation*}
where $\rfun{W}{x}$ is known as the Lambert $W$-function
\cite{weissteinLambert}, satisfying
\begin{equation}
W(x)\ee^{W(x)} = x.
\end{equation}
Transformation of \cref{eq:co-scsc} by \cref{eq:trsf-dtosc} gives
\begin{nameddef}{Regge-Wheeler tortoise coordinates}
The metric of Schwarzschild space-time is given by
\begin{equation}
\dif s^2 = \rbr{1-\frac{\rSch}{r}}\rbr{ -\diff t^2 + \diff {r_*}^2}
+ \rfun{r}{r_*}^2\diff\Omega^2.
\end{equation}
%The range of $r_*$ is $\rbr{-\infty, +\infty}$, corresponding to
%$r \in \rbr{\rSch, +\infty}$.
\end{nameddef}


In order to keep the causal structure in further coordinate transformations,
switch first to the \emph{light-cone} coordinates
\begin{equation}
u = t-r_*\qquad v = t+r_*.
%\\t = \frac{u+v}{2},\quad r_* = \frac{-u+v}{2},
\label{eq:trsf-tlto}
\end{equation}
Using one of them to replace $t$ in \cref{nd:sch-coord} gives
\begin{nameddef}{Eddington-Finklestein coordinates}
The advanced (using $v$) and retarded ($u$) Eddington-Finklestein
coordinates give the forms
\begin{align}
\dif s^2
&= -\rbr{1-\frac{\rSch}{r}}\diff v^2 + 2\diff v\diff r + r^2\diff\Omega^2 \\
&= -\rbr{1-\frac{\rSch}{r}}\diff u^2 - 2\diff u\diff r + r^2\diff\Omega^2 
\end{align}
to the metric, respectively.
%The ranges of $u$ and $r$ are now both $\rbr{-\infty, +\infty}$.
\end{nameddef}
Note that the coordinate singularity at $r=\rSch$ in \cref{nd:sch-coord} has
been eliminated.

whereas the components of the metric are now
\begin{equation}
\dif s^2 = -\rbr{1-\frac{\rSch}{\rfun{r}{u, v}}}\diff u\diff v
+\rfun{r}{u, v}^2\diff\Omega^2.
\label{eq:co-sctlo}
\end{equation}
To bypass the coordinate singularity at $r = \rSch$, using
\cref{eq:trsf-gtosc,eq:trsf-tlto} to get
\begin{equation}
1-\frac{\rSch}{r} = \frac{\rSch}{r}\rfun{\exp}{-\frac{r}{\rSch}}
\rfun{\exp}{\frac{-u+v}{2\rSch}},
\end{equation}
so \cref{eq:co-sctlo} becomes
\begin{equation}
\dif s^2 = -\frac{\rSch}{r}\rfun{\exp}{-\frac{r}{\rSch}}
\rfun{\exp}{\frac{-u+v}{2\rSch}}\diff u\diff v
+ r^2\diff\Omega^2.
\end{equation}
Absorbing the corresponding exponentials by introducing the
\emph{Kruskal-Szekeres} light-cone coordinates,
\begin{equation}
U = -\rSch \rfun{\exp}{-\frac{u}{2\rSch}}
\quad\text{and}\quad
V = +\rSch \rfun{\exp}{+\frac{v}{2\rSch}},
\end{equation}
leads to the new components
\begin{equation}
\dif s^2 = -\frac{4\rSch}{\rfun{r}{U,V}}\rfun{\exp}{-\frac{\rfun{r}{U,V}}{\rSch}}
\diff U\diff V + \rfun{r}{U,V}^2\diff\Omega^2.
\end{equation}
By further rotating the light-cone coordinates into time- and space-like ones,
\begin{equation}
U = T-R\quad\text{and}\quad V = T+R,
\end{equation}
one can derive the Kruskal-Szekeres coordinates, the metric in
which takes the form
\begin{equation}
\dif s^2 = \frac{16\rSch}{\rfun{r}{T,R}}\rfun{\exp}{-\frac{\rfun{r}{T,R}}{\rSch}}
\rbr{-\dif T^2 + \dif R^2} + \rfun{r}{T,R}^2\diff\Omega^2.
\end{equation}

Finally, in order to compactify the coordinates $T$ and $R$ into finite range
while keeping the causal structure, one may introducing the \emph{conformal}
coordinates of light-cone and space-time
\begin{equation}
\mu = \arctan\frac{U}{\rSch} = \eta - \chi \quad\text{and}\quad
\nu = \arctan\frac{V}{\rSch} = \eta + \chi,
\end{equation}
leading to the components
\begin{align}
\dif s^2
&= -\frac{4\rSch^3}{r}\ee^{-r/\rSch}\sec^2\mu\sec^2\nu \diff\mu\diff\nu
+ r^2\diff\Omega^2 \\
&= \frac{16\rSch^3}{r}\ee^{-r/\rSch}
\rfun{\sec^2}{\eta-\chi}\rfun{\sec^2}{\eta+\chi}
\rbr{-\dif\eta^2+\dif\chi^2}
+ r^2\diff\Omega^2.
\end{align}

\subsubsection{Properties}

\begin{namedthm}{Surface gravity}
The surface gravity of Schwarzschild black hole is
\begin{equation}
\kappa_\text{Sch} = \frac{1}{2\rSch}.
\label{eq:sur-gra-sch}
\end{equation}
\end{namedthm}
\begin{proof}
$\partial_t \equiv \rbr{\partial_t}_\text{Sch}$
\begin{align*}
\nabla_{\partial_t}\partial_t &= \rbr{\partial_v}^{\mu'}
\rbr{\nabla_{\mu'}\rbr{\partial_v}^{\nu'}}\partial_{\nu'} \\
&= \delta^{\nu'}_{v} \rbr{\partial_{\nu'} \delta^{\mu'}_{v}
+\Gamma^{\mu'}{}_{\nu'\rho'}\delta^{\rho'}_{v}}\partial_{\nu'} \\
&= \Gamma^{\mu'}{}_{vv}\partial_{\mu'}
= \frac{\rSch}{2r^2}\rbr{\partial_v + \rbr{1-\frac{\rSch}{r}}\partial_r}.
\end{align*}
\begin{equation}
\kappa_\text{Sch} \fat{\partial_t}{\fHor} \equiv
\fat{\nabla_{\partial_t}\partial_t}{\fHor} = \frac{1}{2\rSch} \fat{\partial_t}{\fHor}.
\end{equation}
\end{proof}

\subsection{Other space-times of black hole}

\subsubsection{Spherically symmetric space-times}


\begin{nameddef}{Schwarzschild space-time with cosmological constant}
\begin{equation}
\rfun{f}{r} = 1-\frac{2\nG M}{r} - \frac{\nG \Lambda}{3}r^2.
\end{equation}
\end{nameddef}

Volume

Reissner--Nordström space-time
%\cite{ANDP:ANDP19163550905}

\subsubsection{Axially symmetric space-times}

\begin{nameddef}{Kerr--Newman space-times with cosmological constant}
\begin{equation}
\begin{split}
\dif s^2 = -\frac{\Delta_r}{\Xi^2\varrho^2}
\rbr{\dif t^2-a\sin^2\theta \diff\theta}^2
+ \frac{\varrho^2}{\Delta_r}\diff r^2
+ \frac{\varrho^2}{\Delta_\theta}\diff \theta^2 \\
+ \frac{\Delta_\theta \sin^2\theta}{\Xi^2\varrho^2}
\rbr{a\diff t^2 - \rbr{r^2+a^2}\diff\phi}^2,
\end{split}
\end{equation}
where
\begin{align*}
\varrho^2 &= r^2+a^2\cos^2\theta, \\
\Delta_r &= \rbr{r^2+a^2}\rbr{1-\frac{\nG\Lambda}{3}r^2}-2\nG Mr+\nG Q^2, \\
\Delta_\theta &= 1 + \frac{\nG \Lambda}{3}a^2\cos^2\theta, \\
\Xi &= 1 + \frac{\nG \Lambda}{3} a^2,
\end{align*}
and the potential one-form for the electromagnetic field is
\begin{equation}
\mathrm{A} = -\frac{Qr}{\varrho^2}\rbr{\dif r - a \sin^2 \theta \diff\phi}.
\end{equation}
\end{nameddef}